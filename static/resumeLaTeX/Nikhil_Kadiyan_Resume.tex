\documentclass[]{Nikhil_Kadiyan_Resume}
\usepackage{fancyhdr}
\usepackage{hyperref}
\usepackage{fontawesome5}

\pagestyle{fancy}
\fancyhf{}
\renewcommand{\headrulewidth}{0pt}

\hypersetup{%
 	colorlinks=false,% hyperlinks will be black colour
 	pdfborderstyle={/S/U/W 0.25}% border style will be underline of width 0.25pt
}
 
\begin{document}

%%%%%%%%%%%%%%%%%%%%%%%%%%%%%%%%%%%%%%
%     TITLE NAME
%%%%%%%%%%%%%%%%%%%%%%%%%%%%%%%%%%%%%%
\namesection{Nikhil Kadiyan}{
    % xxx-xxx-xxxx -> Phone Number
	\normalfont {\faMapMarker }Hyderabad, India \hfill {\faAt } \href{mailto:nikhil.kadiyan.20@gmail.com}{nikhil.kadiyan.20@gmail.com} \hfill {\faPhone* } +91 8800475087 \hfill {\faGlobe } \href{https://nikhilcad.github.io/}{Portfolio Website} \hfill {\faLinkedin } \href{https://www.linkedin.com/in/nikhil-kadiyan-3a044b1a0/}{nikhilcad} \hfill {\faGithub } \href{https://github.com/nikhilcad}{nikhilcad}
}

\descript{}

%%%%%%%%%%%%%%%%%%%%%%%%%%%%%%%%%%%%%%
%     EDUCATION
%%%%%%%%%%%%%%%%%%%%%%%%%%%%%%%%%%%%%%
\section{Education}
\hrulefill

\subsection{National Institue of Technology Delhi \hfill \normalfont 2020-24}
\projectposition{Bachelor of Technology (CGPA: 8.44) \normalfont}{ }

\sectionsep

% \subsection{Kendriya Vidyalaya, Shalimar Bagh \hfill \normalfont Delhi, India}
% \position{Class 12 - 94 \% \normalfont}{2020}
% \position{Class 10 - 91.8 \% \normalfont}{2018}

% \sectionsep

%%%%%%%%%%%%%%%%%%%%%%%%%%%%%%%%%%%%%%
%     EXPERIENCE
%%%%%%%%%%%%%%%%%%%%%%%%%%%%%%%%%%%%%%
\section{Experience} 
\hrulefill

\subsection{ThoughtSpot India Pvt Ltd \hfill \normalfont Hyderabad, On-Site}

\position{Member of Technical Staff | Java, Golang, React, Typescript, GraphQL}{Jul '24 - Present}
\pt Achieved a 60\% increase in page load times by migrating the impressions database for the product's homepage from Munshi V1 to V2, achieved by changing the schema of the database\\
\pt Executed the handling of communication events across multiple tenants as measured by the successful implementation of a multi-tenant service (COMS project) by working on email and Slack event handling for all clusters within a tenant. \\
\sectionsep

\position{Software Engineering Intern | React, Typescript, Playwright, Jest}{Jan '24 - Jun `24}
\pt Worked on \textbf{Filter and Sort feature} for worksheet tables and columns along with one team member. Collaborated with the backend team to update API to include \textbf{tags, authors and database info} to sort the tables.\\
\pt Implemented Knowledge Cards that tell users info about a selected table including database. schema, author etc. Made API calls to the backend\\
\pt Migrated schema viewer from Angular to React by creating new React components and making graphQL calls to the backend. Also helped to solve various design issues in old UI during migration\\
\pt Resolved \textbf{high priority bugs} in the existing code on a priority basis for the customer, with average time of 1-2 days to merge the changes into master. \textbf{Cherry picked} must fixes to already released software, added retroactively to older versions of product.\\
\sectionsep

\position{Software Engineering Intern | React, Typescript, Jest}{Jun '23 - Jul '23}
\pt Implemented \textbf{50+ accessibility improvements} in the product's website\\
\pt Worked with \textbf{React.js} and \textbf{Typescript} to ensure website is accessible by \textbf{keyboard and screen readers}, especially for blind users and near-sighted users\\
% \pt Accessibility includes supporting read aloud features
\pt Collaborated with team to ensure that the website is in accordance with \textbf{WAI-ARIA and WCAG} international standards, achieving \textbf{Level A}  accessibility certification\\
\sectionsep


% \subsection{Indian Institue of Technology, Indore \hfill \normalfont Remote}
% \position{Summer Internship in Machine Learning}{Jun '22 - Aug '22}
% \pt Machine Learning, especially regarding \textbf{Hyperspectral Images}\\
% \pt Learnt about various algorithms and concepts like \textbf{Transformers,} \textbf{Neural Network} etc.\\
% \pt Used \textbf{Python and} \textbf{Pytorch }for performing operations like denoising on \textbf{Hyperspectral Images}\\
% \sectionsep

%%%%%%%%%%%%%%%%%%%%%%%%%%%%%%%%%%%%%%
%     PROJECTS
%%%%%%%%%%%%%%%%%%%%%%%%%%%%%%%%%%%%%%
\section{Project Highlights}
\hrulefill

%%%%% SIMPLIFIED FORMAT %%%%%
% \pt Attention-based Joint Detection of Object and Semantic Part in \textbf{PyTorch} \href{https://arxiv.org/abs/2007.02419}{[\textbf{arXiv:2007.02419}]} \href{https://github.com/kevalmorabia97/Object-and-Semantic-Part-Detection-pyTorch}{[\textbf{Code}]}
% \pt Built an application in \textbf{JavaFX} for \textbf{Frequent Patterns \& Association Rule Mining} using Apriori algorithm \href{https://github.com/kevalmorabia97/FPARM-Frequent-Patterns-and-Association-Rule-Miner}{[\textbf{Code}]}

%%%%% ELABORATED FORMAT %%%%%
%\position{Automatic Sanitizing Line Following Robot(4th Semester Project)}{Jun '22}
%\pt Robot capable of \textbf{following an already given path} and sanitizing its surroundings \\
%\pt Robot maintains \textbf{social distancing} by buzzing when two people stand too close \\
%\sectionsep

% \position{Photogram - An image sharing social media platform \href{https://github.com/nikhilCad/photogram}{[Code]} \href{https://photogram-eight.vercel.app/}{[Live]}}{}
% \pt Includes \textbf{user authentication}, \textbf{post and comments creation}, \textbf{home page}, and \textbf{suggested users} to provide an engaging social media experience\\
% \pt Created using \textbf{React} for the frontend, and \textbf{Firebase} for the backend \\
% \pt Uses \textbf{ChakraUI} for UI elements and \textbf{Zustand} for state managements\\
% \sectionsep

% \projectposition{Golang Cloth Management API \href{https://github.com/nikhilCad/go-cloth-management}{[Code]}}{}
% \pt Developed a comprehensive cloth management API in \textbf{Golang} to handle inventory and order management.\\
% \pt Implemented \textbf{JWT token authentication} for secure access and operations within the API.\\
% \pt Created endpoints for updating cloth data and managing customer orders, ensuring efficient and reliable data transactions.\\
% \pt Updated data to a \textbf{MongoDB} database for robust and scalable data storage and retrieval.\\
% \sectionsep

\projectposition{Golang Terminal News Reader \href{https://github.com/nikhilCad/go-news-reader}{[Code]}}{}
\pt Created a terminal news reader in \textbf{Golang} using \textbf{Bubbletea}, for easy navigation of parsed news and YouTube feeds.\\
\pt Integrated \textbf{SQLite} for efficient storage and retrieval of feeds, optimizing data management within the terminal. \\
\pt Implemented content fetching from the URL to \textbf{parse} and display full article text instead of truncated feed contents.\\
\pt Also implemented an \textbf{API} using \textbf{Go} supporting retrieval and addition of feeds and updating them in database\\
\sectionsep

\projectposition{Real Time Chatroom Application \href{https://github.com/nikhilCad/real-time-chatroom}{[Code+Screenshots]} }{}
\pt Implemented a Real Time Chatroom  with \textbf{Next.js} for seamless server side rendering, and \textbf{Clerk} for authentication\\
% \pt Enhanced styling consistency through the application of styles with \textbf{TailwindCSS}  \\
\pt Integrated \textbf{UploadThing} as a robust big data server, supporting file uploads up to 2GB\\
\pt Ensured effective \textbf{CRUD} operations and server communication using the promise-based HTTP library \textbf{Axios}\\
\pt Facilitated real-time bidirectional communication across platforms using \textbf{SocketIO}\\
\sectionsep

\projectposition{Photogram - An image sharing social media platform \href{https://github.com/nikhilCad/photogram}{[Code]} \href{https://photogram-eight.vercel.app/}{[Live]}}{}
\pt Launched Photogram, a dynamic image-sharing platform with \textbf{secure user authentication} using \textbf{Firebase}\\
\pt Implemented \textbf{post, comments, home page and suggested users} functionality for enhanced user engagement \\
\pt Developed responsive UI using \textbf{React},  utilizing \textbf{ChakraUI} for common UI elements\\
\pt Utilized \textbf{Zustand} for efficient state management in the frontend, providing updates on any change in backend\\
\sectionsep

% \projectposition{Created a Clicker Game Using Godot Engine \href{https://github.com/nikhilCad/reality-clicker}{[Code]} \href{https://nikhilcad.itch.io/reality-clicker}{[Live]}}{}
% \pt Built a conventional clicker with various types of \textbf{shops/auto-clicker} upon progression \\
% \pt Has \textbf{exponential difficulty} (similar to other games in genre) to deviate income from in game cost in log scale \\
% \pt Can be played on Windows, Browser and Android, played by 600+ players on \textbf{itch.io}\\
% \pt Has \textbf{Save, Load, and prestige options} to provide engaging gameplay\\
% \sectionsep

% \position{Created a Movie Description Site in React and Flask \href{https://github.com/nikhilCad/flask-movie-app}{[Code]}}{}
% \pt Built a web app that fetches movie title, description and poster using \textbf{Flask} \\
% \pt Uses the \textbf{TMDB API} to get dynamic movie information, using \textbf{React} and \textbf{Node} to build the front-end\\
% \sectionsep

% \position{Created Todos App in React \href{https://github.com/nikhilCad/TodoReactApp}{[Code]} \href{https://nikhilcad.github.io/TodoReactApp}{[Live]}}{}
% \pt Create a simple app for todos using \textbf{React.js} and \textbf{Node.js} \\
% \pt User can \textbf{add or remove task} from the list using the UI
% \sectionsep

% \position{Created my own blog \href{https://github.com/nikhilcad/nikhilcad.github.io}{[Code]} \href{https://nikhilcad.github.io}{[Live]}}{}
% \pt Website created in HTML, CSS and Javascript\\
% \pt Whole project is \textbf{managed on Github using git} \\
% \sectionsep

% \projectposition{Python News Reader App \href{https://github.com/nikhilCad/pySimpleRSS}{[Code+Screenshots]}}{}
% \pt Created a News Reader App using \textbf{Python} \\
% \pt Provided a way to subscribe user's sources using RSS feeds available at preferred website \\
% \pt Use \textbf{beautifulsoup library to parse} the website content with selectors \\
% \sectionsep

% \position{Created game “pyrager” using Python \href{https://github.com/nikhilCad/pyRager}{[Code+Screenshots]}}{}
% \pt Can move using keyboard keys to move. \textbf{Collision detection} using arrays \\
% \pt \textbf{Saving and loading} using Python and text files \\
% \sectionsep

%%%%%%%%%%%%%%%%%%%%%%%%%%%%%%%%%%%%%%
%     SKILLS
%%%%%%%%%%%%%%%%%%%%%%%%%%%%%%%%%%%%%%
\section{Skills} 
\hrulefill

\pt \textbf{Languages:} Python, C++, Java, Go, SQL, TypeScript \\
\pt \textbf{Frameworks:} ReactJS, Jest, Playwright, NextJS, NodeJS, Angular, Flask\\
\pt \textbf{Developer Tools:}  Git, Heroku, Vercel, Firebase, LaTex, VSCode, IntelliJ\\
\pt \textbf{Libraries:} Prisma, Pandas, Geopandas, ChakraUI Bootstrap, rayLib, Godot\\

\sectionsep

%%%%%%%%%%%%%%%%%%%%%%%%%%%%%%%%%%%%%%
%     ACHIEVEMENTS
%%%%%%%%%%%%%%%%%%%%%%%%%%%%%%%%%%%%%%
\section{Achievments} 
\hrulefill

\pt Solved \textbf{400+} Data Structures and Algorithm problems on \textbf{Leetcode} \href{https://leetcode.com/anidnottaken/}{[Link]}, rated \textbf{1545} on contests.\\ 
\pt Achieved a highest Rating of \textbf{1708}, 3 stars in \textbf{Codechef} \href{https://www.codechef.com/users/nikhilcad}{[Link]} \\
\pt Published paper \textbf{Study and Analysis of Plant Disease Identification Models} for ICIC3S Conference, available on \href{https://ieeexplore.ieee.org/document/10603350}{[IEEE Explore]} DOI: 10.1109/ICIC3S61846.2024.10603350\\
% \pt Awarded 5 stars in Python and C on \textbf{Hackerank} \href{https://www.hackerrank.com/nikhilcaddilac?hr_r=1}{[Link]}\\
%\pt \textbf{Other Technologies:} PyTorch, TensorFlow, Scikit-learn, Git, Flutter, Android Studio, ReactJS, Nodejs, Bootstrap, Pandas, Geopandas, Heroku, Godot

\sectionsep

%%%%%%%%%%%%%%%%%%%%%%%%%%%%%%%%%%%%%%
%     LEADERSHIP AND HONORS
%%%%%%%%%%%%%%%%%%%%%%%%%%%%%%%%%%%%%%
%\section{Leadership and Honors} 
%\hrulefill 

%\begin{minipage}[t]{.8\textwidth}
%	\pt Did \textbf{stuff} at 50000-membered as president in my dreams
%\end{minipage}%
%\begin{minipage}[t]{.2\textwidth}
%	\hfill Jul '17 - May '18
%\end{minipage}

%\sectionsep

%%%%%%%%%%%%%%%%%%%%%%%%%%%%%%%%%%%%%%	
\end{document}  \documentclass[]{article}
